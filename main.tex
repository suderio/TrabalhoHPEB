% Created 2024-06-29 sáb 20:28
% Intended LaTeX compiler: pdflatex
\documentclass[article,12pt,oneside,a4paper,english,brazil,sumario=tradicional]{abntex2}
%% \usepackage{lmodern}	
\usepackage[T1]{fontenc}
\usepackage[utf8]{inputenc}
\usepackage{indentfirst}
\usepackage{nomencl}
\usepackage{color}
\usepackage{graphicx}
\usepackage{microtype}
\usepackage[brazilian,hyperpageref]{backref}
\usepackage[alf,bibjustif]{abntex2cite}
\usepackage{fourier}
\usepackage{lipsum}
\renewcommand{\backrefpagesname}{Citado na(s) página(s):~}
\renewcommand{\backref}{}
\renewcommand*{\backrefalt}[4]{ \ifcase #1 Nenhuma citação no texto. \or Citado na página #2. \else Citado #1 vezes nas páginas #2. \fi}
\renewcommand{\ABNTEXsectionfont}{\fontfamily{ptm}\fontseries{b}\selectfont}

\needspace{3cm}
\tituloestrangeiro{}
\data{2024}
\local{Brasil}
\orientador{XXXXXX xXXXX}
\instituicao{%
  Pontifícia Universidade Católica do Paraná -- PUCPR
  \par
  Pós-Graduação em História e Relações Internacionais
  \par
  Programa de Pós-Graduação}
\tipotrabalho{Projeto de Pesquisa}

\definecolor{blue}{RGB}{41,5,195}
\makeatletter
\makeatother
\makeindex
\setlrmarginsandblock{3cm}{3cm}{*}
\setulmarginsandblock{3cm}{3cm}{*}
\checkandfixthelayout
\setlength{\parindent}{1.25cm}
\setlength{\parskip}{0.2cm}  % tente também \onelineskip
\author{Aguinaldo Jorge da Silva \and Denerval \and Diogo Teixeira de Andrade Ferreira Gomes \and Matheus Dell Antonio \and Paulo Alípio Suderio Rodrigues}
\date{2024}
\title{RELAÇÕES ENTRE BRASIL E A RÚSSIA E AS CONSEQUÊNCIAS PARA O AGRONEGÓCIO NO BRASIL}
\hypersetup{
 pdftitle={RELAÇÕES ENTRE BRASIL E A RÚSSIA E AS CONSEQUÊNCIAS PARA O AGRONEGÓCIO NO BRASIL}, 
 pdfauthor={Paulo Alípio Suderio Rodrigues},
 pdfsubject={},
 pdfcreator={Emacs 29.4 (Org mode 9.6.15)},
 pdfkeywords={},
 pdflang={English},
 colorlinks=true,
 linkcolor=blue,
 citecolor=blue,
 filecolor=magenta,
 urlcolor=blue,
 bookmarksdepth=4}

\begin{document}
\selectlanguage{brazil}
\frenchspacing
\maketitle
\textual
\section*{Tema Escolhido}

O conflito entre a Rússia e a Ucrânia abalou o comércio internacional pois os Estados Unidos e a União Europeia aplicaram sanções. 

O Brasil mesmo não aderindo a sanções comerciais aplicada a Rússia, sofre pressões dos Estados Unidos por não condenar enfaticamente a invasão.

Neste contexto, o Brasil importa diversos produtos e em especial o petróleo e derivados e fertilizantes.

A Rússia é uma grande fornecedora de fertilizantes e adubos para o Brasil. Sendo o Brasil um dos grandes parceiros que mais importa esses produtos, por necessitar em grande escala desses bens para a produção agrícola.

Neste contexto, a presente pesquisa demonstra a relevância da avaliação do cenário atual do agronegócio em detrimento da guerra entre a Rússia e a Ucrânia. 

\section*{Problema de Pesquisa}

Quais os impactos econômicos que o Brasil poderá sofrer com o conflito da Rússia e a Ucrânia? O Brasil poderá sofrer alguma advertência pelos Estados Unidos e União Europeia se continuar comprando produtos russos, tais como derivado do petróleo e fertilizantes? As politicas norte americanas de isolar a Rússia poderá afetar a economia brasileira de continuar e apoiando a Rússia?

\section*{Hipóteses}

A Guerra Russa-Ucraniana prejudica os preços de barril de petróleo e de seus derivados, incluindo fertilizantes, com impacto assimétrico no Brasil.

A dependência do Brasil por fertilizantes, principalmente da Rússia, obriga a busca pelo produto em outros países durante a duração do conflito.

Há consequências econômicas, políticas e geopolíticas tanto na aproximação (ou leniência) com Rússia quanto no alinhamento ao bloco ocidental.


\section*{Justificativa}

A importância do projeto deste artigo é entender a continuidade das relações entre o Brasil e a Rússia no contexto do conflito do leste europeu  e por outro lado a influência norte- americana e da União Europeia para o Brasil aderir as sanções da Rússia.

O projeto irá ponderar a busca por produtos em especial os fertilizantes e o petróleo e seus derivados da Rússia que estão com preços mais baixos, não podendo perder de vista as advertências dos Estados Unidos neste contexto da Guerra Russa Ucraniana.

\section*{Objetivos}

\subsection*{Objetivo Geral}

Analisar a posição do Brasil frente as suas relações comerciais com a Rússia com as importações de petróleo e fertilizantes.

\subsection*{Objetivos Específicos}

Analisar se a importação destes produtos sob o prisma de possíveis retaliações dos Estados Unidos e da União Europeia.

\section*{Metodologia}

O presente trabalho será desenvolvido através de pesquisa bibliográfica.

\section*{Estrutura}

\renewcommand\thesubsection{\arabic{subsection}}

\subsection{Introdução}

A avaliação da relação entre Brasil e Rússia após o início da guerra na Ucrânia exige uma análise cuidadosa de diversos elementos, particularmente no que diz respeito ao comércio internacional e à dependência brasileira de insumos agrícolas importados, como os fertilizantes. Esse contexto não só reflete as implicações de uma realpolitik, onde interesses econômicos moldam decisões diplomáticas, mas também revela nuances complexas do impacto da guerra sobre os fluxos comerciais globais e os interesses nacionais estratégicos.

A relação Brasil-Rússia pós-guerra na Ucrânia envolve uma tensão entre pragmatismo econômico e compromissos diplomáticos. A interdependência comercial no setor de fertilizantes destaca a importância de uma política externa que maximize a autonomia estratégica do Brasil enquanto protege o desempenho econômico do agronegócio. A longo prazo, o Brasil deve equilibrar a neutralidade e a diversificação de suas relações comerciais e políticas, preservando sua inserção competitiva no sistema internacional e sua capacidade de garantir segurança alimentar.

\subsection{Relação entre Brasil e Rússia}

\subsection{Políticas Comerciais}

Desde a independência, o Brasil optou por manter relações comerciais com a Europa e os EUA, em detrimento de outros países, o que inviabilizou o comércio com a Rússia. Essa situação começou a se reverter apenas em 1963, quando o desejo por uma política externa independente guiava as decisões do país, possibilitando a assinatura do Acordo de Comércio e Pagamentos entre a URSS e o Brasil, que visava facilitar as trocas comerciais. Após o colapso do regime comunista em Moscou, abriram-se diversas possibilidades comerciais, que foram se intensificando paulatinamente por meio de iniciativas como os BRICS. No entanto, a maior fase de aproximação comercial ocorreu após a anexação da Crimeia, quando a Rússia buscou novos parceiros comerciais devido às sanções impostas, estreitando, a partir de 2017, as relações com o Brasil. Esse estreitamento é evidenciado pela declaração conjunta sobre o Diálogo Estratégico em Política Externa, pelo Plano de Consultas Políticas e pelo Memorando de Entendimento na Área de Cooperação Econômica e de Investimentos. Em síntese, o aspecto mais evidente nessa longa trajetória de política comercial entre ambos é a importância que os fertilizantes russos possuem para o Brasil, com a Rússia sendo o maior fornecedor desses insumos ao mercado brasileiro, especialmente potássio e ureia, produtos essenciais para a agricultura intensiva brasileira. No entanto, atualmente existe profundas preocupações, pois, devido à Guerra na Ucrânia, as novas sanções impostas à Rússia como represália afetam diretamente o futuro dessa relação, bem como expõem a dependência brasileira da importação desses recursos.

\subsection{A Importância do Adubo Fertilizante}

O uso de fertilizantes no Brasil é fundamental para a alta produtividade do agronegócio, especialmente para a soja, milho e cana-de-açúcar, onde a demanda por nutrientes essenciais é elevada. Os fertilizantes são compostos, principalmente, de três macronutrientes: nitrogênio, potássio e fósforo, que exercem funções essenciais no desenvolvimento das plantas. O nitrogênio é essencial para o crescimento vegetal, promovendo a formação de folhas e contribuindo para a fotossíntese. Já o fósforo auxilia para a floração e frutificação das plantas. E por ultimo o potássio é responsável pelo aumento da eficiência da saúde da planta. O Brasil, no entanto, depende fortemente da importação desses insumos, adquirindo mais de 80\% dos fertilizantes utilizados internamente. Antes da Guerra na Ucrânia, a Rússia era o principal fornecedor de fertilizantes ao mercado brasileiro, especialmente de potássio. Em 2021, cerca de 28\% dos fertilizantes importados pelo Brasil vieram da Rússia.Com o início do conflito e a imposição de sanções à Rússia, o Brasil enfrentou desafios para manter o fornecimento regular desses produtos, o que evidenciou a necessidade de diversificação dos fornecedores e do fortalecimento da produção interna. Essas medidas são parte do Plano Nacional de Fertilizantes, que visa garantir a segurança e estabilidade do agronegócio, vital para a economia brasileira, reduzindo gradualmente a dependência de insumos importados. 

\subsection{Guerra Russa-Ucraniana}

No dia 24 de fevereiro de 2022, a Rússia invadiu a Ucrânia, após meses de tensões e ameaças na fronteira russo-ucraniana.

Uns dos motivos da invasão seria a pretensão da Ucrânia, em aderir na União Europeia e também Organização do Tratado do Atlântico Norte (OTAN) tendo sido incluídas na constituição no ano de 2019, tendo a Rússia se opondo a entrada na Otan, alegando que seria uma ameaça a ela. (Wikipédia, 2024)

Esse conflito bélico iniciado pelas forças militares russas em território ucraniano é o acontecimento com impacto geopolítico global mais significativo das últimas décadas (RAMOS et al. 2022).

Os primeiros impactos que surgiram no Brasil, logo após o início da guerra, foram econômicos. A guerra subiram os preços dos combustíveis e da energia – a Rússia, afinal, é o maior exportador mundial de gás natural e o segundo maior exportador de petróleo (SCHOSSLER, 2023).

Como o aumento da energia e dos fertilizantes subiram os preços dos produtos alimentícios, o que pode gerar diminuição na produção agrícola. A alta dos preços de alguns grãos também aumentou o custo com ração animal de janeiro a abril de 2022. Na pecuária de corte, o custo elevou para 51,7\% para cria, 72,2\% para recria e engorda e 76\% para confinamento (REVISTA SEGUROS, 2023).

Por fim, neste cenário o Brasil sofreu muita pressão  para não importar produtos russos, pois foi sancionado pelos Estados Unidos e União Europeia pela invasão da Ucrânia.  

\subsection{Conclusão}
 
   A Rússia é um dos principais fornecedores de fertilizantes para o agronegócio brasileiro, especialmente de potássio e outros insumos críticos para a produção agrícola intensiva. Com a guerra, sanções ocidentais e instabilidade na cadeia logística russa, emergiu um cenário de volatilidade e insegurança em relação à oferta e ao preço desses produtos. Dada a importância do agronegócio na economia brasileira, essa dependência representa um ponto de vulnerabilidade: interrupções no fornecimento poderiam impactar a produção, influenciar preços internos de alimentos e até mesmo comprometer a balança comercial.

   O agronegócio é um dos pilares do superávit da balança comercial brasileira, sendo que a soja, o milho e outras commodities agrícolas têm mercado global e estão diretamente atrelados à produtividade agrícola sustentada pelo uso de fertilizantes. O Brasil, portanto, enfrenta um dilema estratégico. Por um lado, a continuidade das importações de fertilizantes russos é essencial para manter a competitividade do setor. Por outro, o alinhamento com políticas de sanções lideradas por países ocidentais poderia limitar ou interromper esse fluxo, afetando a estabilidade econômica e as relações comerciais.

   Com a guerra, o Brasil identificou a necessidade de reduzir a dependência de um único fornecedor de insumos estratégicos. Nesse sentido, uma diversificação das fontes de importação de fertilizantes tornou-se prioridade na agenda governamental e empresarial. Países como o Canadá, o Marrocos e até investimentos internos no setor mineral brasileiro ganharam relevância como alternativas, embora tal mudança exija tempo e investimento para se concretizar de forma viável.

   Desde o início do conflito, o Brasil adotou uma postura diplomática cautelosa, evitando um posicionamento condenatório direto à Rússia, optando por uma linha de neutralidade que reforça a histórica política externa de autonomia e não intervenção. Essa neutralidade permite ao Brasil preservar laços comerciais com a Rússia, evitando consequências econômicas diretas para setores sensíveis. Contudo, essa posição é objeto de críticas e pressões, especialmente de parceiros ocidentais, que esperam alinhamento com suas diretrizes políticas e econômicas.

   A guerra também trouxe à tona o papel do BRICS como um bloco potencialmente alinhado com uma visão multipolar e, por vezes, de resistência à hegemonia ocidental. Para o Brasil, o BRICS, do qual a Rússia também é membro, representa uma plataforma para fortalecer relações econômicas e políticas com países não alinhados com a OTAN, permitindo que se distancie das pressões geopolíticas sem romper laços com potências ocidentais. Essa dinâmica, porém, é complexa, e uma ênfase excessiva na Rússia no contexto do BRICS pode implicar riscos diplomáticos caso a postura de neutralidade seja vista como complacente.


% \nocite{pd2003john}
% \nocite{castro2012teoria}

\newpage

\postextual

\bibliography{global}

%% \glossary

%% \begin{apendicesenv}
%% \chapter{Nullam elementum urna vel imperdiet sodales elit ipsum pharetra ligula ac pretium ante justo a nulla curabitur tristique arcu eu metus}
%% \lipsum[55-56]
%% \end{apendicesenv}

%% \cftinserthook{toc}{AAA}
%% \anexos
%% \begin{anexosenv}
%% \chapter{Cras non urna sed feugiat cum sociis natoque penatibus et magnis dis
%% parturient montes nascetur ridiculus mus}
%% \lipsum[31]
%% \end{anexosenv}

% \section{Agradecimentos}
% \label{sec:org40c5706}
% Texto sucinto aprovado pelo periódico em que será publicado. Último elemento pós-textual.
\end{document}

